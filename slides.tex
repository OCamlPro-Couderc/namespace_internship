\documentclass{beamer}
\usepackage[utf8]{inputenc}
\usepackage{listings}
\usepackage{syntax}
\usepackage{cite}
\usepackage{color}

\newsavebox{\simpleimport}

\setbeamercovered{transparent}

\author{Pierrick COUDERC, sous la direction de Fabrice Le Fessant 
\\
Inria}

\title{Master 2 STL - Soutenance de stage \\
  -\\
  Namespaces pour OCaml}

\date{8 Septembre 2014}

%% \setlength{\grammarindent}{8em} % increase separation between LHS/RHS 

\lstnewenvironment{OCaml}
                  {\lstset{
                      language=[Objective]Caml,
                      breaklines=true,
                      %% commentstyle=\color{purple},
                      stringstyle=\color{red},
                      identifierstyle=\ttfamily,
                      keywordstyle=\color{blue},
                      basicstyle=\footnotesize\tt,
                      otherkeywords={namespace, abstract}
                    }
                  }
                  {}

\begin{document}

\maketitle

\begin{frame}
\frametitle{Problématique: utiliser des modules de même nom}

\only<1-2>{
On suppose la situation suivante:

\begin{itemize}
\item LibA: (Misc, Map, AnotherModule, ...) 
\item Mon programme: (Misc, Env, ...)
\end{itemize}

\bigskip

}

\only<2>{
\textbf{ocamlopt} : \emph{``Error: Files libA/misc.cmx and misc.cmx both define a module named Misc''}
}

\only<3>{
\begin{itemize}
\item LibA (\alert{Misc}, Map, ...) 
\item Mon programme (\alert{Misc}, Env, ...)
\end{itemize}
}

\only<4-6>{
\begin{itemize}
\item LibA (Misc, Map, ...) 
\item Mon programme (\alert{My}Misc, Env, ...)
\end{itemize}

\bigskip
}

\only<5-6>{

\textbf{ocamlopt} : \emph{``Error: Files libA/anotherModule.cmx and env.cmx
  make inconsistent assumptions over interface Map''}
%% \emph{``Hey, you use Map in Env but AnotherModule also use Map, and they're
%% not the same!''}

\bigskip
}

\only<6>{
$\rightarrow$ Impossible d'utiliser Map de la stdlib
}

\only<7-9>{
Quelles solutions ?
\begin{itemize}
\item<7-> Copier les sources de map dans mon programme ?
\item<8-> Abandonner la stdlib ?
\item<9-9> Demander au développeur de renommer son module ?
\end{itemize}
}

\end{frame}

\begin{frame}[fragile]
\frametitle{Avant 4.02: solution \#1}

Pratique classique : l'utilisation de \emph{noms longs}

\bigskip

\only<2->{
\begin{itemize}
\item<2-> LibA (Misc, Map, ...)
\item<3-> $\rightarrow$ LibA (\alert{LibA}\_Misc, \alert{LibA}\_Map, ...)
\end{itemize}

\bigskip
}

\only<4>{
\begin{itemize}
\item Obligation de tout renommer
\item Utilisation de plusieurs bibliothèques $\rightarrow$ relativement verbeux
\end{itemize}
}

\end{frame}

\begin{frame}[fragile]
\frametitle{Avant 4.02: solution \#2}

Utilisation de \emph{packs de modules}

\bigskip

\only<2-3>{
\begin{itemize}
\item<2-3> LibA (Misc, Map, ...)
\item<3-3> $\rightarrow$ LibA = un \alert{seul module} LibA
  
  contenant pour \alert{sous-modules}: (Misc, Map, ...)
\end{itemize}

\bigskip
}

\only<4->{
Avantages:
\begin{itemize}
\item<4->\textbf{Developpeur}: Pas de modification du code, simplement des
  options du compilateur
\item<5->\textbf{Utilisateur}: 
  \begin{itemize}
  \item Possibilité d'utiliser le chemin \texttt{Pack.Module}
  \item Possibilité d'utiliser le pack comme n'importe quel module
  \end{itemize}
\end{itemize}
\medskip
}

\only<6->{
Inconvénients:
\begin{itemize}
\item<6-> Arbre de recompilation très important
\item<7-> Dépendances difficilement calculables
\item<8-> Exécutables produits particulièrement gros
\end{itemize}
}

\end{frame}

\begin{frame}[fragile]
\frametitle{Avec 4.02: solution \#3}

\defverbatim[colored]{\exalias}{
\begin{OCaml}
  (* libA.ml *)
  module Misc = LibA_Misc
  module Map = LibA_Map
  ...
\end{OCaml}
}

Introduction des \emph{alias de modules}

\bigskip

\only<2->{
\begin{itemize}
\item<3-> LibA (Misc, Map, ...)
\item<4-> $\rightarrow$ LibA = (LibA\_Misc, LibA\_Map, ...) \alert{+ LibA}

\exalias

compilé avec \textbf{-no-alias-deps}:

$\rightarrow$ Pas nécessaire de \emph{linker} les alias non utilisés
\end{itemize}

}

\end{frame}

\begin{frame}[fragile]
\frametitle{Simuler des namespaces à l'aide d'alias}

\defverbatim[colored]{\exalias}{
\begin{OCaml}
(* libA_misc.ml *)
open LibA
...
\end{OCaml}

\begin{OCaml}
(* libA_map.ml *)
open LibA

open Misc
...
\end{OCaml}}

\bigskip

\only<2-3>{
Intérêt pour le développeur: utiliser aussi les noms courts

\exalias
}

\only<3-3>{
\bigskip

Circularité $LibA \prec Map \land Map \prec Lib$ ?

$\rightarrow$ Non, grâce à \textbf{-no-alias-deps}
}

\defverbatim[colored]{\exaliasbis}{
\begin{OCaml}
(* map.ml *)
open Misc
...
\end{OCaml}
}

\only<4-5>{
\begin{itemize}
\item<4-> ``Tricher'' pour des dépendances correctes

\exaliasbis

\item<5> \textbf{+} Utiliser le namespace de manière transparente

$\rightarrow$

\textbf{ocamlc -c -o libA_Map.cmo -open LibA map.ml}
\end{itemize}
}

\end{frame}

\begin{frame}
\frametitle{Solutions actuelles}
\begin{itemize}[<+->]
\item Dépendences difficiles à calculer
\item Nécessite un bon système de \emph{build}
\item Pas d'extensibilité : ajout de modules \emph{a posteriori} impossible.
\end{itemize}
\end{frame}

\begin{frame}
\frametitle{Solution proposée}
La solution: un méchanisme de \emph{namespaces}.

\bigskip

Principe:
\begin{itemize}[<+->]
\item Chaque module appartient à un namespace
\item Les modules sont explicitement importés
\end{itemize}

\bigskip

\only<3>{
  Utilisation d'un en-tête pour gérer les namespaces

  Le principe est similaire aux packages de Java ou Scala.
}
\end{frame}

\begin{frame}[fragile]
\frametitle{Réécrire LibA avec des namespaces}

Comment réécrire LibA?

\pause

\begin{columns}%% [t]{5cm}
  \column{0.6\textwidth}
    \begin{OCaml}
      (* misc.ml *)
      in namespace LibA
      ...
    \end{OCaml}

  \column{0.6\textwidth}
    \begin{OCaml}
      (* map.ml *)
      in namespace LibA
      ...
    \end{OCaml}
\end{columns}

\end{frame}

\begin{frame}[fragile]
\frametitle{Utiliser la bibliothèque}

\only<1>{
Comment utiliser les modules du namespaces ?
}

\defverbatim[colored]{\simpleimport}{
\begin{OCaml}
in namespace MyNamespace
with MyNamespace.Misc
and LibA.(Misc, Map)

open Misc (* from LibA *)
let empty = Map.empty
...
let _ = ... 
  Misc.pprint 42 (* function only in my own Misc *)
\end{OCaml}}

\only<2-3>{\simpleimport \bigskip}

\only<3>{
\textbf{ocamlc}: \emph{``Unbound value Misc.pprint''}}

\defverbatim[colored]{\aliasedimport}{
\begin{OCaml}
in namespace MyNamespace
with MyNamespace.Misc
and LibA.(Misc as Misc2, Map)

open Misc2
let empty = Map.empty
...
let _ = ... 
  Misc.pprint 42
\end{OCaml}}

\only<4->{
  Il faut importer à la fois Misc et ``Misc depuis LibA'':
  
  \bigskip

  \aliasedimport
}

\end{frame}

\begin{frame}[fragile]
\frametitle{Utilisation de tous les modules}

Importer tous les modules :

\bigskip

\defverbatim[colored]{\wildcard}{
\begin{OCaml}
in namespace MyNamespace
with LibA._

open Misc (* from LibA *)
...
\end{OCaml}}

\only<3>{
  \wildcard

\bigskip

Non recommandé car masque des modules précédemment importés.

}

\end{frame}

\begin{frame}[fragile]
\frametitle{Presque tout utiliser}

Importer tous les modules, sauf certains :

\bigskip

\defverbatim[colored]\wildcardcstr{
\begin{OCaml}
in namespace MyNamespace
with MyNamespace._
and LibA.(Misc as _, _)
...
open Misc (* not the Misc from LibA *)
\end{OCaml}
}


\only<2-3>{\wildcardcstr

\bigskip

\`{A} noter:
\begin{itemize}
\item<2-3> Aucune dépendance vers LibA.Misc : le module n'est pas seulement
  caché, il n'apparait pas dans l'environnement.
\item<3-3> Non recommandé (pour les mêmes raisons).
\end{itemize}}

\end{frame}

\begin{frame}[fragile]
\frametitle{Conflit de noms}

Si deux noms indentiques sont utilisés:

\bigskip

\defverbatim[colored]\warning{
\begin{OCaml}
in namespace MyNamespace
with Stdlib._
and LibA._
...
\end{OCaml}
}

\only<2-3>{
\warning
\bigskip}

\only<3>{
\textbf{ocamlc}: \emph{``The module Map from LibA will shadow one previously imported''}

}

\end{frame}

\begin{frame}[fragile]
\frametitle{Extensibilité}
Au contraire des modules, les namespaces ne sont pas \emph{clos}.

\bigskip

\defverbatim[colored]{\extens}{
\begin{OCaml}
in namespace LibA
...
\end{OCaml}
}

\only<2>{\extens
\bigskip
Insérer un module dans un namespace \emph{a posteriori} (qui existe donc déjà) est possible}
\end{frame}

\begin{frame}[fragile]
\frametitle{Etendre Pervasives aux namespaces}

\only<1-2>{
Par défaut, Pervasives est automatiquement ouvert.
}

\defverbatim[colored]{\pervasives}{
\begin{OCaml}
in namespace MyNamespace
with Stdlib.List
...
let empty = [] (* from Stdlib.Pervasives *)
\end{OCaml}}

\only<2>{
\pervasives

\bigskip

$\rightarrow$ par conséquent, Pervasives d'un namespace est ouvert si ce dernier
est utilisé.
}

\defverbatim[colored]{\palias}{
\begin{OCaml}
with Stdlib.(Pervasives as P)
\end{OCaml}
}

\defverbatim[colored]{\pshadow}{
\begin{OCaml}
with Stdlib.(Pervasives as _)
\end{OCaml}
}

\only<3-4>{
Empêcher l'ouverture automatique: 
\bigskip
\begin{itemize}
\item<3-> En renommant:
\palias
\item<4-> Ou en masquant:
\pshadow
\end{itemize}
}
\end{frame}

\begin{frame}[fragile]
\frametitle{Organisation hiérarchique}
Namespaces : organisation naturelle de modules (comme une bibliothèque).

\medskip
\pause

Une possible organisation de Stdlib:
\begin{OCaml}
with Stdlib.Internals.CamlinternalFormat
and Stdlib.Unsafe.Obj
...
\end{OCaml}
\end{frame}

\begin{frame}
\frametitle{Namespaces et système de fichiers}

A la compilation: un namespace $\Rightarrow$ un dossier

\medskip

\pause

\begin{itemize}[<+->]
\item Logique : permet de distinguer les unités de même nom
\item Pratique : organisation automatique par le compilateur
\item Future utilisation : permettre une option \emph{-make} similaire à GHC
\end{itemize}

\end{frame}

\begin{frame}
\frametitle{Calcul des dépendances}

Déclaration et imports de namespaces $\equiv$ en-tête de fichier

\bigskip

Simplification du calcul de dépendances: chaque module \textbf{est forcément} un fichier

\bigskip

Intérêt : calcul syntaxique des dépendances
$\rightarrow$ ocamldep peut ne lire que l'en-tête, sans \emph{typer} le code

\end{frame}

%% \begin{frame}
%% \frametitle{Formal and technical aspect}

%% Namespaces, especially imports:

%%  $\rightarrow$ Description of the compilation environnement

%% \bigskip

%% Compiler-side: not too invasive
%% \begin{itemize}[<+->]
%% \item Symbols extended to contain the namespace
%% \item Env extended to use and propagate namespaces
%% \end{itemize}

%% \end{frame}

\begin{frame}
\frametitle{Alias face à notre solution}
\begin{itemize}[<+->]
\item \textcolor{blue}{\textbf{+}} Extensibilité
\item \textcolor{blue}{\textbf{+}} Makefile ou systèmes de build simples
\item \textcolor{blue}{\textbf{+}} Meilleure calcul des dépendances
\item \textcolor{blue}{\textbf{+}} Plus expressif
\item \textcolor{red}{\textbf{-}} Nouvelle syntaxe $\rightarrow$ le code n'est
  pas compatible avec des compilateurs plus anciens
\end{itemize}
\end{frame}

\begin{frame}[fragile]
\frametitle{Works in progress: Coercion vers modules}

Idée : transformer l'en-tête en un module

\begin{OCaml}
with LibA.(Misc, Map)
and Stdlib.(List, String, Map)
\end{OCaml}

$$\Rightarrow$$

\pause

\begin{OCaml}
module LibA = struct
  module Misc = ...
  module Map = ...
end
module Stdlib = struct
  module List = ...
  module String = ...
  module Map = ...
end
\end{OCaml}

%% \pause

%% \begin{itemize}
%% \item Full path for modules
%% \item Namespaces as first-order modules
%% \item Namespaces as functor arguments
%% \end{itemize}

\end{frame}

\begin{frame}
\frametitle{Work in progress: \emph{big functors}}

A la base : travail de Fabrice Le Fessant pour générer des foncteurs dans les packs

\bigskip

Exemple: Cohttp $\rightarrow$ utilise énormément de foncteurs pour être
paramétrique et utiliser à la fois Lwt et Async

\bigskip

Principe :

$\Rightarrow$ Générer ces foncteurs automatiquement sur l'ensemble du namespace,
et l'appliquer tardivement

\medskip

Expérimental, avec des choix arbitraires à revoir.

\end{frame}

\begin{frame}[fragile]
\frametitle{Conclusion}

\begin{itemize}
\item Véritable méchanisme de namespaces dans le langage
\item Résoud des problèmes de compilation
\item Grand intérêt pour le calcul de dépendances
\item Prototype fonctionnel (OCaml 4.02):

\texttt{github.com/pcouderc/ocaml_namespaces}
\end{itemize}

\end{frame}

\end{document}
