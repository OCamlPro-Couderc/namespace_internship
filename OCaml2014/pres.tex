\documentclass{beamer}
\usepackage[utf8]{inputenc}
\usepackage{listings}
\usepackage{syntax}
\usepackage{cite}

\author{Pierrick COUDERC, Fabrice LE FESSANT, Benjamin Canou, Pierre Chambart}

\title{A Proposal for Non-intrusive Namespaces for OCaml}

%% \setlength{\grammarindent}{8em} % increase separation between LHS/RHS 

\lstnewenvironment{OCaml}
                  {\lstset{
                      language=[Objective]Caml,
                      breaklines=true,
                      %% commentstyle=\color{purple},
                      %% stringstyle=\color{red},
                      %% identifierstyle=\ttfamily,
                      %% keywordstyle=\color{blue},
                      %% basicstyle=\footnotesize\tt,
                      otherkeywords={namespace, abstract}
                    }
                  }
                  {}

\begin{document}

\maketitle

\begin{frame}
\frametitle{Introduction}

Namespaces are a recurrent discussion amongst the community. However, no
agreement emerged on what they are, why are they useful and how to add them in
OCaml.

\medskip

What we present is a proposal with a prototype of working namespace mechanism.
\end{frame}

\begin{frame}[fragile]
\frametitle{On the need of namespaces}

\defverbatim{\ex}{
\begin{OCaml}
  (* mylib.ml *)
  module List = Mylib_list
  module Map = Mylib_map
  ...
\end{OCaml}
}


Currently, no way to distinguish two compilation units of the same name, thus to
link them.

Solutions: 
\begin{itemize}
\only<1>{\item Hopefully long enough names, for example ``My\_lib\_utils'' or
  ``Core\_list''}
\only<2>{\item Using \emph{packs} to avoid long names}
\only<3>{\item Using a module which is simply a list of aliases:}
\end{itemize}
\only<3>{\ex1}
\end{frame}

%% \begin{frame}
%% \frametitle{On the need of namespaces}


%% \end{frame}

\begin{frame}
\frametitle{What namespaces are}

In the current OCaml standard, there is no notion of namespaces, or
packages. They do not exist at the language level.

\medskip

Namespaces are a meaning to distinguish two modules (compilation units) with the
same name. Moreover, they are also a standard solution to install libraries.

\medskip

Basically namespaces are a solution to express libraries inside the language and
not only on the build system side. 

\end{frame}

\begin{frame}[fragile]
\frametitle{Writing Mylib with namespaces}

If we want to write the same modules from previous examples ?

\pause

%% \defverbatim{\extwo}{
\begin{columns}%% [t]{5cm}
  \column{0.6\textwidth}
    \begin{OCaml}
      (* list.ml *)
      in namespace Mylib
      ...
    \end{OCaml}

  \column{0.6\textwidth}
    \begin{OCaml}
      (* map.ml *)
      in namespace Mylib
      ...
    \end{OCaml}
\end{columns}
%% }

and the results are \textbf{mylib/list.cmo} and \textbf{mylib/map.cmo}.
\end{frame}

\begin{frame}[fragile]
\frametitle{Using our library}

Both modules List and Map from Mylib cannot be used as standard modules. They
must be imported into the environment.

\pause

\begin{OCaml}
with (List; Map) of Mylib
...
\end{OCaml}

\pause

But we also want to use them with the module List of the stdlib:

\begin{OCaml}
with (List as Mylist; Map) of Mylib
...
\end{OCaml}

\end{frame}

\begin{frame}[fragile]
\frametitle{Import possibilities}

Importing every modules one by one from a namespace can be extremely painful. As
many languages, importing \textbf{all} of them is possible:

\begin{OCaml}
with (_) of Batteries
...
\end{OCaml}

\pause

However, you can choose which module(s) \textbf{not} to import:
\begin{OCaml}
with (_; List as _; Hashtbl as _) of Batteries
and (List as Mylist) of Mylib
and (Hashtbl) of OCamlPro.Stdlib
...
\end{OCaml}

\end{frame}

\begin{frame}
\frametitle{Namespaces and imports as header}

Namespace declaration can only happen at the beginning of the file, before any
declaration. Importing namespaced modules is only possible after this
declaration.

\medskip

As a result, namespaces do not exists inside the core language, they are only a
meaning to describe the environment of compilation. This header simply adds the
modules into the environment, therefore this mechanism \textbf{does not change}
the type system.

\end{frame}

\begin{frame}[fragile]
\frametitle{Namespaces and filesystem}
In other languages (Java or Scala in particular), namespaces (as packages) are
tightly related. In this namespace proposal, namespaces are directories on the
filesystem.

\medskip

For example, every modules in the namespace Ocamlpro.Stdlib can be found (from
the loadpath) in \texttt{ocamlpro/stdlib}.

\medskip

This solution could harmonize how libraries are installed on the filesystem, and
would help the compiler and build systems to find compilation units more efficiently.

\end{frame}

\begin{frame}
\frametitle{Big functors}

Big functors are an extension of packs, where every modules are wrapped inside a
functor. 

\medskip

Since packs where a workaround for namespaces, big functors (or functor-packs)
where a solution for parametric libraries (or parametric namespaces).

\end{frame}

\begin{frame}
\frametitle{Big functors in the context of namespaces}
\end{frame}

\end{document}
