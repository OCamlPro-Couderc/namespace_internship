\documentclass{beamer}
\usepackage[utf8]{inputenc}
\usepackage{listings}
\usepackage{syntax}
\usepackage{cite}


\setbeamercovered{transparent}

\author{Pierrick COUDERC, Fabrice LE FESSANT, Benjamin CANOU, Pierre CHAMBART}

\title{A Proposal for Non-intrusive Namespaces for OCaml}

%% \setlength{\grammarindent}{8em} % increase separation between LHS/RHS 

\lstnewenvironment{OCaml}
                  {\lstset{
                      language=[Objective]Caml,
                      breaklines=true,
                      %% commentstyle=\color{purple},
                      %% stringstyle=\color{red},
                      %% identifierstyle=\ttfamily,
                      %% keywordstyle=\color{blue},
                      %% basicstyle=\footnotesize\tt,
                      otherkeywords={namespace, abstract}
                    }
                  }
                  {}

\begin{document}

\maketitle

\begin{frame}
\frametitle{Introduction}

Namespaces are a recurrent discussion amongst the community. However, no
agreement emerged on what they are, why are they useful and how to add them in
OCaml.

\medskip

What we present is a proposal with a prototype of working namespace mechanism.
\end{frame}

\begin{frame}[fragile]
\frametitle{On the need of namespaces}

\defverbatim{\ex}{
\begin{OCaml}
  (* mylib.ml *)
  module List = Mylib_list
  module Map = Mylib_map
  ...
\end{OCaml}
}


Currently, no way to distinguish two compilation units of the same name, thus to
link them.

Solutions: 
\begin{itemize}
\only<1>{\item Hopefully long enough names, for example ``My\_lib\_utils'' or
  ``Core\_list''}
\only<2>{\item Using \emph{packs} to avoid long names}
\only<3>{\item Using a module which is simply a list of aliases:}
\end{itemize}
\only<3>{\ex}
\end{frame}

%% \begin{frame}
%% \frametitle{On the need of namespaces}


%% \end{frame}

\begin{frame}
\frametitle{What namespaces are}

In the current OCaml standard, there is no notion of namespaces, or
packages. They do not exist at the language level.

\medskip

Namespaces are a meaning to distinguish two modules (compilation units) with the
same name. Moreover, they are also a standard solution to install libraries.

\medskip

Basically namespaces are a solution to express libraries inside the language and
not only on the build system side. 

\end{frame}

\begin{frame}[fragile]
\frametitle{Writing Mylib with namespaces}

If we want to write the same modules from previous examples?

\pause

%% \defverbatim{\extwo}{
\begin{columns}%% [t]{5cm}
  \column{0.6\textwidth}
    \begin{OCaml}
      (* list.ml *)
      in namespace Mylib
      ...
    \end{OCaml}

  \column{0.6\textwidth}
    \begin{OCaml}
      (* map.ml *)
      in namespace Mylib
      ...
    \end{OCaml}
\end{columns}
%% }

and the results are \textbf{mylib/list.cmo} and \textbf{mylib/map.cmo}.
This namespace declaration must be the first statement of the module.
\end{frame}

\begin{frame}[fragile]
\frametitle{Using our library}

Both modules List and Map from Mylib cannot be used as standard modules. They
must be imported into the environment.

\pause

\begin{OCaml}
with (List; Map) of Mylib
...
\end{OCaml}

\pause

But we also want to use them with the module List of the stdlib:

\begin{OCaml}
with (List as Mylist; Map) of Mylib
...
\end{OCaml}

\end{frame}

\begin{frame}[fragile]
\frametitle{Import possibilities}

Importing every modules one by one from a namespace can be extremely painful. As
many languages, importing \textbf{all} of them is possible:

\begin{OCaml}
with (_) of Batteries
...
\end{OCaml}

\pause

However, you can choose which module(s) \textbf{not} to import:
\begin{OCaml}
with (_; List as _; Hashtbl as _) of Batteries
...
\end{OCaml}

\end{frame}

\begin{frame}
\frametitle{Namespaces and imports as header}

Namespace declaration can only happen at the beginning of the file, before any
declaration. Importing namespaced modules is only possible after this
declaration.

\medskip

As a result, namespaces do not exists inside the core language, they are only a
meaning to describe the environment of compilation. This header simply adds the
modules into the environment, therefore this mechanism \textbf{does not change}
the type system.

\end{frame}

\begin{frame}[fragile]
\frametitle{Namespaces and filesystem}
In other languages (Java or Scala in particular), namespaces (as packages) are
tightly related. In this namespace proposal, namespaces are directories on the
filesystem.

\medskip

For example, every modules in the namespace Ocamlpro.Stdlib can be found (from
the loadpath) in \texttt{ocamlpro/stdlib}.

\medskip

This solution could harmonize how libraries are installed on the filesystem, and
would help the compiler and build systems to find compilation units more efficiently.

\end{frame}

\begin{frame}
\frametitle{Advantages}
\begin{itemize}[<+>]
\item Avoid recompilation with simple build system (GNU Make) (bottlenecks
  with packs and module aliases);
\item Allows linking of compilation unit with the same name;
\item No change in the current semantics;
\item Basically some kind of ``long names'', more expressive thanks to import
  constraints.
\end{itemize}
\end{frame}

\begin{frame}
\frametitle{Drawbacks}
\begin{itemize}[<+>]
\item At compile time, the compiler cannot create directories;
\end{itemize}
\end{frame}

\begin{frame}
\frametitle{Big functors}

Big functors are an extension of packs, where every modules are wrapped inside a
functor. 

\medskip

Since packs were a workaround for namespaces, big functors are a workaround for
parametric namespaces.

\end{frame}

\begin{frame}[fragile]
  \frametitle{Simplest solution for parametric namespaces}
    \begin{OCaml}
      (* myHttpLib/requestHandler.ml *)
      in namespace MyHttpLib
      module Make (Arg : AsyncMonad) = 
        struct 
          ...
        end
    \end{OCaml}

    \pause 

    \begin{OCaml}
      (* myHttpLib/protocol.ml *)
      in namespace MyHttpLib
      module Make (Arg : AsyncMonad) = 
        struct
          module RequestHandler = RequestHandler.Make(Arg)
          ...
        end
    \end{OCaml}
\end{frame}

\begin{frame}
  \frametitle{Drawbacks of this solution}
  \begin{itemize}[<+->]
    \item Multiple instantiation of each module

      $\rightarrow$ Each module must instantiate its dependencies to use them;
    \item No sharing: references are local to each instance, initialisation code
      computed each time
    
      $\rightarrow$ Terrible performance.
  \end{itemize}
\end{frame}

\begin{frame}
  \frametitle{Big functors in the context of namespaces}

  We introduce the concept of \textbf{functor units}, i.e. compilation units that
  are functorized over some interfaces. 

  \medskip

  \begin{itemize}
    \item Only on the build system side;
    \item Application results on a new compilation unit;
    \item Each functor instantiated \alert{once};
    \item When applying the functor-unit, we look for already applied
      dependencies.
  \end{itemize}

\end{frame}

\begin{frame}[fragile]
\frametitle{Compiling a functor unit}

\only<1>{

  Suppose we write an HTTP library (as previous example) that uses an
  asynchronous monadic combinator to handle requests: there exist multiple
  libraries that are usable (Async, Lwt).

}

\defverbatim{\exmonad}{
  \begin{OCaml}
    (* myHttpLib/asyncMonad.mli *)
    in namespace MyHttpLib

    type 'a t

    val (>>=) : 'a t -> ('a -> 'b t) -> 'b t

    val return : 'a -> 'a t
    val bind : 'a t -> ('a -> 'b t) -> 'b t
    val poll : 'a t -> 'a option
    ...
  \end{OCaml}
}

\only<2>{
  First, we declare the signature of this monad:
  \exmonad
}

\only<3>{

  We suppose we have multiple files in our library that uses this monad:
  \begin{itemize}
  \item myHttpLib/requestHandler.ml
  \item myHttpLib/protocol.ml
  \item ...
  \end{itemize}

}

\only<4>{
  Each are compiled with the following command:
  
  \begin{block}{}
    \texttt{ocamlc -c -functor asyncMonad.cmi <filename>.ml}
  \end{block}
  \bigskip

  As a result, each compilation unit in \texttt{myHttpLib} are functorized over
  the \texttt{AsyncMonad} module. The namespace \texttt{MyHttpLib} is parametric
  over this particular interface.
}

\end{frame}

\begin{frame}[fragile]
\frametitle{Instantiating functor units}

\only<1> {
  A functor unit must be instantiated to be used. Arguments must be given to the
  funtor unit to generate a correct compilation unit.
}

\defverbatim{\lwtbinding}{
  \begin{OCaml}
    (* myHttpLib/Lwt/lwt.ml *)

    in namespace MyHttpLib.Lwt
    with (Lwt) of Ocsigen

    include Lwt
  \end{OCaml}
}

\only<2> {
  The argument we use for this instantiation:
  \medskip
  \lwtbinding
}

\only<3> {
  This module can be applied on each of the units:
  
  \begin{block}{}
    \texttt{ocamlc -c -apply (...)/lwt.cmi (...)/<unit>.cmo}
  \end{block}

  \medskip

  Each of those applied modules will belong to the namespace
  \texttt{MyHttpLib.Lwt} $\equiv$ same namespace as the argument. 
}

\end{frame}


%% \begin{frame}
%% \frametitle{Functor units: properties}

%% \begin{itemize}
%% \item<1> The argument of the functor unnit must belong to the same namespace
%% \item<2> The instantiated modules belongs to the same namespace as the argument
%%   used at application
%% \item<3> Different instantiations can be used at the same time.
%% \item<4> This solution is expressible in the core language by wrapping each
%%   module in a functor that instantiates each of its dependencies. In his
%%   configuration, the build system ease the job.
%% \end{itemize}

%% \end{frame}

\begin{frame}
\frametitle{Functor units: restrictions}
\begin{itemize}[<+->]
\item Each cmi given to \texttt{-functor} must belong to the same namespace as
  the functor unit currently compiled;
\item Functor units can have dependencies:
  \begin{itemize}
    \item Standard compilation units
    \item Functor units with the \alert{same arguments}
    \item Their arguments
  \end{itemize}
\item No currification
\item Compunits used for application must belong to the same namespace.
\end{itemize}
\end{frame}

\begin{frame}
\frametitle{Prototype}

Each feature introduced is implemented in a prototype based on the 4.02
revision of the compiler.

Available at ... 
\end{frame}

\begin{frame}
\frametitle{Conclusion and future work}

Our solution may be not the best, its purpose is to revive the discussions with
a proposal and a working prototype. 

\medskip

Future work:
\begin{itemize}
  \item Cleaning and refactoring the code
  \item For functor-units:
    \begin{itemize}
      \item Add currification
      \item Leverage the dependencies restrictions:

        a functor unit can depend on functor units whose arguments are a subset
        of its own.
    \end{itemize}
  \item Community ideas
\end{itemize}
\end{frame}

\end{document}
